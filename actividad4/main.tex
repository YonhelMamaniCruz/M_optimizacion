\documentclass{beamer}
\usetheme{Madrid}

\title{Un modelo de programación lineal entera mixta para la cosecha, carga y transporte de caña de azúcar}
\subtitle{Un estudio de caso en el Perú}
\author{Yonhel Mamani Cruz}
\date{\today}

\begin{document}

% Diapositiva 1: Portada con introducción breve
\begin{frame}
  \titlepage
\end{frame}

\begin{frame}{ Optimización de la cosecha, carga y transporte de caña de azúcar}
  \begin{itemize}
    \item El documento aborda un problema real en Perú: cómo organizar eficientemente la \textbf{cosecha, carga y transporte} de caña de azúcar desde los campos hasta las fábricas.
    \vspace{0.3cm}
    \item El objetivo es \textbf{minimizar los costos logísticos} y \textbf{cumplir restricciones operativas}, como:
    \begin{itemize}
      \item Ventanas de tiempo para la cosecha y entrega.
      \item Disponibilidad limitada de equipos y recursos humanos.
      \item Capacidad de transporte y balance de carga hacia las plantas.
    \end{itemize}
  \end{itemize}
\end{frame}

% Diapositiva 2: ¿Qué es MILP?
\begin{frame}{Fundamentos del modelo de optimización}
  \begin{itemize}
    \item \textbf{Programación lineal entera mixta (MILP)}: combina variables continuas (como toneladas) y enteras (como número de vehículos).
    \item Ideal para representar decisiones logísticas:
    \begin{itemize}
      \item Qué, cuándo, dónde y cómo cosechar, cargar y transportar.
    \end{itemize}
    \item Se usa MILP porque el problema tiene combinaciones complejas de decisiones con múltiples restricciones.
  \end{itemize}
\end{frame}

% Diapositiva 3: Estructura del modelo
\begin{frame}{Componentes clave del modelo matemático}
  \textbf{Variables de decisión:}
  \begin{itemize}
    \item Si un campo es cosechado un día específico.
    \item Cantidad de caña transportada por camión.
    \item Asignación de equipos y horarios.
  \end{itemize}

  \vspace{0.3cm}
  \textbf{Función objetivo:}
  \begin{itemize}
    \item Minimizar los costos totales de transporte, cosecha y penalizaciones por retraso.
  \end{itemize}

  \vspace{0.3cm}
  \textbf{Restricciones:}
  \begin{itemize}
    \item Capacidad de camiones y equipos.
    \item Ventanas de cosecha por campo.
    \item Balance de carga en fábricas.
    \item Disponibilidad de recursos diarios.
  \end{itemize}
\end{frame}

% Diapositiva 4: Aplicación en Perú
\begin{frame}{Cómo se aplicó MILP en el estudio de caso}
  \begin{itemize}
    \item \textbf{Datos reales de una empresa azucarera en Perú:}
    \begin{itemize}
      \item 11 campos de cultivo
      \item 2 fábricas
      \item 3 frentes de cosecha
      \item 4 camiones
    \end{itemize}
    \item El modelo permitió generar un plan diario óptimo de cosecha y transporte.
    \item Se usó software especializado para resolver el modelo (\textbf{CPLEX}).
    \item Se obtuvo una reducción significativa de costos comparado con la planificación manual.
  \end{itemize}
\end{frame}

% Diapositiva 5: Conclusión
\begin{frame}{Beneficios de usar MILP en logística agrícola}
  \begin{itemize}
    \item Mayor eficiencia en uso de recursos.
    \item Reducción de tiempos muertos y costos.
    \item Planeación más precisa y realista.
    \item Este tipo de modelos puede replicarse en otras agroindustrias para mejorar procesos logísticos y tomar mejores decisiones.
  \end{itemize}
\end{frame}

\end{document}





