\documentclass{article}

% Language setting
% Replace `english' with e.g. `spanish' to change the document language
\usepackage[english]{babel}

% Set page size and margins
% Replace `letterpaper' with `a4paper' for UK/EU standard size
\usepackage[letterpaper,top=2cm,bottom=2cm,left=3cm,right=3cm,marginparwidth=1.75cm]{geometry}

% Useful packages
\usepackage{amsmath}
\usepackage{graphicx}
\usepackage[colorlinks=true, allcolors=blue]{hyperref}

\title{Tarea 1; SUEÑO, variables, funciones y restriccion}
\author{Yonhel Mamani Cruz}

\begin{document}
\maketitle

\begin{abstract}
definiciones
\end{abstract}

\section{Introduccion}

En los articulos e investigaciones que lei y del cual extrai informacion, el sueño normal se define como un estado de disminución de la conciencia y de la posibilidad de reaccionar frente a los estímulos que nos rodean. Es un estado reversible -lo cual lo diferencia de otras condiciones patológicas como el coma-, y se presenta con una periodicidad cercana a las 24 horas o “circadiana”. Tras años de estudios no ha decaído entre los investigadores, básicos y clínicos, el interés por desentrañar todos los aspectos de este fenómeno. Aunque persisten muchas incógnitas, podemos afirmar que es imprescindible para la vida

\section{DEFINICION: Variable, Funcion y Restriccion}

\subsection{Variable; Horas de sueño}

Horas de sueño es una variable cuantitativa continua que representa la cantidad de tiempo, medida en horas, que una persona duerme en un período de 24 horas. Esta variable puede variar de un individuo a otro y de un día a otro, y se utiliza comúnmente para estudiar la relación entre el descanso y diversas funciones biológicas, cognitivas y emocionales.

\subsection{funcion; Rendimiento cognitivo}

Una función es una relación matemática que asigna a cada valor de una variable independiente (por ejemplo, las horas de sueño) un único valor de una variable dependiente (por ejemplo, el rendimiento cognitivo, la concentración o el estado de ánimo).

En el contexto del sueño, una función permite modelar cómo cambia una respuesta biológica o mental en función de cuántas horas duerme una persona.

\subsection{Restriccion; un limite logico y biologico }

Una restricción es una condición o límite impuesto sobre los valores que puede tomar una variable dentro de un modelo matemático. En el contexto del sueño, una restricción establece los valores mínimos y/o máximos que pueden asumir las horas de sueño debido a factores biológicos, sociales o de salud.

\subsection{Ejemplo aplicado}

Vamos a realizar un ejemplo aplicado a la vida real y cotidiana. Este ejemplo va acompañado con las investigaciones realizadas por la universidad san martin de porres, por el autor miguel silva lopez.

Una variable: 
x = horas de sueño por noche

Una función:
que relacione el rendimiento cognitivo con las horas de sueño

Una restricción:
que imponga un límite lógico o biológico

Ejemplo:
Variable:
x= número de horas de sueño por noche

Función:
El rendimiento cognitivo (R) en función de las horas de sueño:

R(x)=−(x−8)2+100

Esta es una parábola invertida, con máximo en x=8 horas (óptimas), y 
R=100 es el rendimiento máximo. Si duermes más o menos, el rendimiento baja.

Restricción:
El cuerpo humano no puede dormir menos de 4 ni más de 12 horas regularmente sin consecuencias graves. Entonces:

4 menor o igual que x menor o igual que 12

Interpretación:
Si duermes 8 horas:

R(8)=-(8-8)2+100=100
→ Máximo rendimiento.

Si duermes 5 horas:

R(5)=-(5-8)2+100=-9+100=91
→ Tu rendimiento baja. 



\bibliographystyle{alpha}
\bibliography{sample}
ARTICULOS E INVESTIGACIONES del cual extraje informacion

https://repositorio.usmp.edu.pe/bitstream/handle/20.500.12727/2570/SILVA_M.pdf?sequence=1

Revista Médica Clínica Las Condes

Miró, Elena; Cano Lozano, María del Carmen; Buela Casal, Gualberto
Sueño y calidad de vida
Revista Colombiana de Psicología, núm. 14, 2005, pp. 11-27
Universidad Nacional de Colombia
Bogotá, Colombia

\end{document}